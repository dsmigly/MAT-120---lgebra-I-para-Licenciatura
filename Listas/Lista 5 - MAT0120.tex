\documentclass[12pt, a4paper]{article}
\usepackage[bottom=2cm,top=3cm,left=2cm,right=2cm]{geometry}
\usepackage[portuguese]{babel}
\usepackage[utf8]{inputenc}
\usepackage{CJKutf8}
\usepackage{mathtext}
\usepackage{wrapfig}
\usepackage[T1]{fontenc}
\usepackage{blindtext}
\usepackage{tasks}
\usepackage{setspace}
\usepackage{verbatim}
\usepackage[dvipsnames]{xcolor}
%\usepackage{tikz}
\usepackage{tikz-cd}%Para diagramas 
\usepackage[framemethod=Tikz]{mdframed}
\usepackage{amsmath}
\usepackage{amsfonts}
\usepackage{amssymb}
\usepackage{wasysym}
\usepackage{amsthm}
\usepackage{graphics}
\usepackage{pifont}
\usepackage{arydshln} %dashed line nas matrizes
%\usepackage{lipsum}
%\usepackage{CJKutf8} %Pacote para escrever em japonês \begin{CJK}{UTF8}{min} \end{CJK}
\usepackage{multicol}
% \usepackage{colorspace}
\usepackage[arrow,matrix,curve]{xy}
\usepackage{enumitem}
\usepackage{graphicx, color}
%\usepackage{eulervm} %Fonte de texto
\usepackage{exsheets} %Mostrar soluções
\usepackage{commath}
\usepackage{mathpazo}
\usepackage{cancel}
%-------------------------------------------------------------
%Comandos úteis
\newcommand{\mdc}{{\rm mdc}}
\newcommand{\mmc}{{\rm mmc}}
\newcommand{\sen}{{\rm sen}}
\newcommand{\tg}{{\rm tg}}
\newcommand{\cotg}{{\rm cotg}}
\newcommand{\cossec}{{\rm cossec}}
\newcommand{\arctg}{{\rm arctg}}
\newcommand{\arcsen}{{\rm arcsen}}
\newcommand{\negrito}[1]{\mbox{\boldmath{$#1$}}} 
\newcommand{\heart}{\ensuremath\heartsuit}
\newcommand{\diamonde}{\ensuremath\diamondsuit}
\newtheorem{defi}{Definição}
\newtheorem{prop}{Proposição}
\newtheorem{dem}{Demonstração}
\newtheorem{coro}{Corolário}
\DeclareSymbolFont{extraup}{U}{zavm}{m}{n}
\DeclareMathSymbol{\varheart}{\mathalpha}{extraup}{86}
\DeclareMathSymbol{\vardiamond}{\mathalpha}{extraup}{87}
\setlength{\parindent}{0pt}
\newcommand{\tn}[1]{\textnormal{#1}} 
\newcommand{\Hom}{\tn{Hom}}
\newcommand{\End}{\tn{End}}
\newcommand{\Z}{\mathbb{Z}}
\newcommand{\R}{\mathbb{R}}
\renewcommand{\rmdefault}{ptm} 
\newcommand{\spec}{\textrm{Spec}}

%-------------------------------------------------------------
%Boxes para critérios de correção, caso seja necessário

%Alternativa Verde e azul para avisos    
    \mdfdefinestyle{Criterios}{%
    linecolor=blue,
    outerlinewidth=2pt,
    roundcorner=10pt,
    innertopmargin=\baselineskip,
    innerbottommargin=\baselineskip,
    innerrightmargin=20pt,
    innerleftmargin=20pt,
    backgroundcolor=white!75!green}
    
%Box padrão    
\mdfdefinestyle{MyFrame}{%
    linecolor=blue,
    outerlinewidth=2pt,
    roundcorner=20pt,
    innertopmargin=\baselineskip,
    innerbottommargin=\baselineskip,
    innerrightmargin=20pt,
    innerleftmargin=20pt,
    backgroundcolor=white!50!white}
    
%\mdfdefinestyle{Solução}{%
%    linecolor=blue,
%    outerlinewidth=1pt,
%    roundcorner=8pt,
%    innertopmargin=4pt%\baselineskip,
%    innerbottommargin=0pt%\baselineskip,
%    innerrightmargin=20pt,
%    innerleftmargin=20pt,
%    backgroundcolor=white!50!white}
    
%Alternativa Verde e azul para avisos    
    \mdfdefinestyle{Aviso}{%
    linecolor=blue,
    outerlinewidth=2pt,
    roundcorner=20pt,
    innertopmargin=\baselineskip,
    innerbottommargin=\baselineskip,
    innerrightmargin=20pt,
    innerleftmargin=20pt,
    backgroundcolor=white!50!green}
    
%----------------------------------------------------------------
%Cores do documento
\definecolor{Floresta}{rgb}{0.13,0.54,0.13}    
% \definespotcolor{mygreen}{PANTONE 7716 C}{.83, 0, .00, .51}
% \definespotcolor{tuti}{}{0.6, 0, 1, .508}

%----------------------------------------------------------------
%Uso de counters para numeração automática dos exercícios 
%Para mais infos: https://www.overleaf.com/learn/latex/Counters

\newcounter{exercicio}[section]
\newenvironment{exercicio}[1][]{\refstepcounter{exercicio}\par\medskip
 \textcolor{blue}{\bf(\theexercicio)} \rmfamily}{\medskip }
   
 \usepackage{ifmtarg}% http://ctan.org/pkg/ifmtarg
 
 \makeatletter 
 \newcommand{\isempty}[1]{%
  \@ifmtarg{#1}{\begin{question}}{\begin{question}[topic=#1]}}

% \newcommand{\isempty}[1]{%
%  \@ifmtarg{#1}{\begin{question}}{\begin{question}[topic=#1]}}
  
    
 
 
\newenvironment{solucao}[1][]{\textbf{\\ \\ \textcolor{red}{Solução:}}#1 \rmfamily}{\medskip }


\newenvironment{criterios}[1][]{
 \textcolor{blue}{\bf Critérios de Correção:} \rmfamily \par\medskip #1 }{\medskip }
 
\newcommand{\itens}[1]{\begin{tasks}[label={(tsk[a])},label-width=3.6ex, label-format = {\bfseries}, column-sep = {0pt}](1) #1\end{tasks}}

\newcommand{\itensladoalado}[2]{\begin{tasks}[label={(tsk[a])},label-width=3.6ex, label-format = {\bfseries}, column-sep = {0pt}](#1) #2 \end{tasks}}

\newcommand{\alt}[1]{\textcolor{Floresta}{$\negrito{(#1)} $}}

%\newcommand{\alt}[1]{\task[\pers{#1}]}

%\newcommand{\solucao}[1]{
%\textbf{\\ \\ \textcolor{red}{Solução:}} #1}

%----------------------------------------------------------------
%Dados da lista:

\newcommand{\titulo}{MAT0120 - Álgebra I para Licenciatura}
\newcommand{\lista}{Lista 5}
\newcommand{\professor}{Kostiantyn Iusenko}
\newcommand{\monitor}{Douglas de Araujo Smigly}
\newcommand{\semestre}{1º Semestre de 2021}
%----------------------------------------------------------------

%Mostrar ou não as soluções
%\SetupExSheets{solution/print=true} %Se print=false, este arquivo Não imprime as soluções.
\SetupExSheets{solution/print=false}


%Solução sem numeração (pois vem logo depois da questão) - se quiser com números, basta comentar os comandos abaixo:
\DeclareInstance{exsheets-heading}{block-no-nr}{default}{
  attach = {
    main[l,vc]title[l,vc](0pt,0pt) ;
    main[r,vc]points[l,vc](\marginparsep,0pt)
  }
}

%  {question}[headings=block-subtitle] - por bloco
\RenewQuSolPair
  {question}[headings=block-no-nr]
  {solution}[headings=block-no-nr]

%Personalização de Questão/ Solução
\SetupExSheets{
  counter-format=se.qu.,
  question/name= ,
  solution/name=\textcolor{red}{Solução}
}

%Cabeçalho
%----------------------------------------------------------------
\title{\vspace{-15mm}\fontsize{16pt}{10pt}\selectfont\textbf{\titulo} \\ \vspace{5mm} \textbf{\textcolor{Floresta}{\lista}} \PrintSolutionsT{ - \textcolor{blue}{Soluções}}} % Article title
%\title{\fontsize{16pt}{10pt}{\textbf{MAT0501/MAT6680 - Tópicos de Anéis e Módulos}}
\author{Professor: \professor \\ Monitor: \monitor}
\date{\semestre}
\begin{document}
\maketitle
%------------------------------------------------------------
%Caso queira multicols, só descomentar as linhas abaixo 
%e lembrar de colocar \end{multicols*} no final do documento
%e columnbreak se quiser criar uma nova coluna

%\begin{multicols*}{2}
%\setlength{\columnseprule}{0.78pt}
%\raggedcolumns
%\columnbreak
%------------------------------------------------------------

%------------------------------------------------------------
%
% Exemplos de exercícios:
%
%------------------------------------------------------------
\section{Inteiros módulo $m$}
%Questão simples
\begin{exercicio}
Construa as tabelas de adição e de multiplicação de $\mathbb Z_7$ e $\mathbb Z_{12}$.
\end{exercicio}
\begin{solution}
Para $\mathbb{Z}_7,$ temos as seguintes tabelas:
\begin{center}
\begin{table}[h]
\begin{tabular}{ccccccccccccccccc}
\textbf{$+$}             & $0$                      & $1$                      & $2$                      & $3$                      & $4$                      & $5$                      & $6$                      &  & \textbf{$\cdot$}         & $0$                      & $1$                      & $2$                      & $3$                      & $4$                      & $5$                      & $6$                      \\ \cline{2-8} \cline{11-17} 
\multicolumn{1}{c|}{$0$} & \multicolumn{1}{c|}{$0$} & \multicolumn{1}{c|}{$1$} & \multicolumn{1}{c|}{$2$} & \multicolumn{1}{c|}{$3$} & \multicolumn{1}{c|}{$4$} & \multicolumn{1}{c|}{$5$} & \multicolumn{1}{c|}{$6$} &  & \multicolumn{1}{c|}{$0$} & \multicolumn{1}{c|}{$0$} & \multicolumn{1}{c|}{$0$} & \multicolumn{1}{c|}{$0$} & \multicolumn{1}{c|}{$0$} & \multicolumn{1}{c|}{$0$} & \multicolumn{1}{c|}{$0$} & \multicolumn{1}{c|}{$0$} \\ \cline{2-8} \cline{11-17} 
\multicolumn{1}{c|}{$1$} & \multicolumn{1}{c|}{$1$} & \multicolumn{1}{c|}{$2$} & \multicolumn{1}{c|}{$3$} & \multicolumn{1}{c|}{$4$} & \multicolumn{1}{c|}{$5$} & \multicolumn{1}{c|}{$6$} & \multicolumn{1}{c|}{$0$} &  & \multicolumn{1}{c|}{$1$} & \multicolumn{1}{c|}{$0$} & \multicolumn{1}{c|}{$1$} & \multicolumn{1}{c|}{$2$} & \multicolumn{1}{c|}{$3$} & \multicolumn{1}{c|}{$4$} & \multicolumn{1}{c|}{$5$} & \multicolumn{1}{c|}{$6$} \\ \cline{2-8} \cline{11-17} 
\multicolumn{1}{c|}{$2$} & \multicolumn{1}{c|}{$2$} & \multicolumn{1}{c|}{$3$} & \multicolumn{1}{c|}{$4$} & \multicolumn{1}{c|}{$5$} & \multicolumn{1}{c|}{$6$} & \multicolumn{1}{c|}{$0$} & \multicolumn{1}{c|}{$1$} &  & \multicolumn{1}{c|}{$2$} & \multicolumn{1}{c|}{$0$} & \multicolumn{1}{c|}{$2$} & \multicolumn{1}{c|}{$4$} & \multicolumn{1}{c|}{$6$} & \multicolumn{1}{c|}{$1$} & \multicolumn{1}{c|}{$3$} & \multicolumn{1}{c|}{$5$} \\ \cline{2-8} \cline{11-17} 
\multicolumn{1}{c|}{$3$} & \multicolumn{1}{c|}{$3$} & \multicolumn{1}{c|}{$4$} & \multicolumn{1}{c|}{$5$} & \multicolumn{1}{c|}{$6$} & \multicolumn{1}{c|}{$0$} & \multicolumn{1}{c|}{$1$} & \multicolumn{1}{c|}{$2$} &  & \multicolumn{1}{c|}{$3$} & \multicolumn{1}{c|}{$0$} & \multicolumn{1}{c|}{$3$} & \multicolumn{1}{c|}{$6$} & \multicolumn{1}{c|}{$2$} & \multicolumn{1}{c|}{$5$} & \multicolumn{1}{c|}{$1$} & \multicolumn{1}{c|}{$4$} \\ \cline{2-8} \cline{11-17} 
\multicolumn{1}{c|}{$4$} & \multicolumn{1}{c|}{$4$} & \multicolumn{1}{c|}{$5$} & \multicolumn{1}{c|}{$6$} & \multicolumn{1}{c|}{$0$} & \multicolumn{1}{c|}{$1$} & \multicolumn{1}{c|}{$2$} & \multicolumn{1}{c|}{$3$} &  & \multicolumn{1}{c|}{$4$} & \multicolumn{1}{c|}{$0$} & \multicolumn{1}{c|}{$4$} & \multicolumn{1}{c|}{$1$} & \multicolumn{1}{c|}{$5$} & \multicolumn{1}{c|}{$2$} & \multicolumn{1}{c|}{$6$} & \multicolumn{1}{c|}{$3$} \\ \cline{2-8} \cline{11-17} 
\multicolumn{1}{c|}{$5$} & \multicolumn{1}{c|}{$5$} & \multicolumn{1}{c|}{$6$} & \multicolumn{1}{c|}{$0$} & \multicolumn{1}{c|}{$1$} & \multicolumn{1}{c|}{$2$} & \multicolumn{1}{c|}{$3$} & \multicolumn{1}{c|}{$4$} &  & \multicolumn{1}{c|}{$5$} & \multicolumn{1}{c|}{$0$} & \multicolumn{1}{c|}{$5$} & \multicolumn{1}{c|}{$3$} & \multicolumn{1}{c|}{$1$} & \multicolumn{1}{c|}{$6$} & \multicolumn{1}{c|}{$4$} & \multicolumn{1}{c|}{$2$} \\ \cline{2-8} \cline{11-17} 
\multicolumn{1}{c|}{$6$} & \multicolumn{1}{c|}{$6$} & \multicolumn{1}{c|}{$0$} & \multicolumn{1}{c|}{$1$} & \multicolumn{1}{c|}{$2$} & \multicolumn{1}{c|}{$3$} & \multicolumn{1}{c|}{$4$} & \multicolumn{1}{c|}{$5$} &  & \multicolumn{1}{c|}{$6$} & \multicolumn{1}{c|}{$0$} & \multicolumn{1}{c|}{$6$} & \multicolumn{1}{c|}{$5$} & \multicolumn{1}{c|}{$4$} & \multicolumn{1}{c|}{$3$} & \multicolumn{1}{c|}{$2$} & \multicolumn{1}{c|}{$1$} \\ \cline{2-8} \cline{11-17} 
\end{tabular}
\end{table}
\end{center}
Para $\mathbb{Z}_{12},$ temos as seguintes tabelas:
\begin{center}
    \begin{table}[h]
\begin{tabular}{ccccccccccccc}
\textbf{$+$}              & $0$                       & $1$                       & $2$                       & $3$                       & $4$                       & $5$                       & $6$                       & \multicolumn{1}{l}{$7$}   & \multicolumn{1}{l}{$8$}   & \multicolumn{1}{l}{$9$}   & \multicolumn{1}{l}{$10$}  & \multicolumn{1}{l}{$11$}  \\ \cline{2-13} 
\multicolumn{1}{c|}{$0$}  & \multicolumn{1}{c|}{$0$}  & \multicolumn{1}{c|}{$1$}  & \multicolumn{1}{c|}{$2$}  & \multicolumn{1}{c|}{$3$}  & \multicolumn{1}{c|}{$4$}  & \multicolumn{1}{c|}{$5$}  & \multicolumn{1}{c|}{$6$}  & \multicolumn{1}{c|}{$7$}  & \multicolumn{1}{c|}{$8$}  & \multicolumn{1}{c|}{$9$}  & \multicolumn{1}{c|}{$10$} & \multicolumn{1}{c|}{$11$} \\ \cline{2-13} 
\multicolumn{1}{c|}{$1$}  & \multicolumn{1}{c|}{$1$}  & \multicolumn{1}{c|}{$2$}  & \multicolumn{1}{c|}{$3$}  & \multicolumn{1}{c|}{$4$}  & \multicolumn{1}{c|}{$5$}  & \multicolumn{1}{c|}{$6$}  & \multicolumn{1}{c|}{$7$}  & \multicolumn{1}{c|}{$8$}  & \multicolumn{1}{c|}{$9$}  & \multicolumn{1}{c|}{$10$} & \multicolumn{1}{c|}{$11$} & \multicolumn{1}{c|}{$0$}  \\ \cline{2-13} 
\multicolumn{1}{c|}{$2$}  & \multicolumn{1}{c|}{$2$}  & \multicolumn{1}{c|}{$3$}  & \multicolumn{1}{c|}{$4$}  & \multicolumn{1}{c|}{$5$}  & \multicolumn{1}{c|}{$6$}  & \multicolumn{1}{c|}{$7$}  & \multicolumn{1}{c|}{$8$}  & \multicolumn{1}{c|}{$9$}  & \multicolumn{1}{c|}{$10$} & \multicolumn{1}{c|}{$11$} & \multicolumn{1}{c|}{$0$}  & \multicolumn{1}{c|}{$1$}  \\ \cline{2-13} 
\multicolumn{1}{c|}{$3$}  & \multicolumn{1}{c|}{$3$}  & \multicolumn{1}{c|}{$4$}  & \multicolumn{1}{c|}{$5$}  & \multicolumn{1}{c|}{$6$}  & \multicolumn{1}{c|}{$7$}  & \multicolumn{1}{c|}{$8$}  & \multicolumn{1}{c|}{$9$}  & \multicolumn{1}{c|}{$10$} & \multicolumn{1}{c|}{$11$} & \multicolumn{1}{c|}{$0$}  & \multicolumn{1}{c|}{$1$}  & \multicolumn{1}{c|}{$2$}  \\ \cline{2-13} 
\multicolumn{1}{c|}{$4$}  & \multicolumn{1}{c|}{$4$}  & \multicolumn{1}{c|}{$5$}  & \multicolumn{1}{c|}{$6$}  & \multicolumn{1}{c|}{$7$}  & \multicolumn{1}{c|}{$8$}  & \multicolumn{1}{c|}{$9$}  & \multicolumn{1}{c|}{$10$} & \multicolumn{1}{c|}{$11$} & \multicolumn{1}{c|}{$0$}  & \multicolumn{1}{c|}{$1$}  & \multicolumn{1}{c|}{$2$}  & \multicolumn{1}{c|}{$3$}  \\ \cline{2-13} 
\multicolumn{1}{c|}{$5$}  & \multicolumn{1}{c|}{$5$}  & \multicolumn{1}{c|}{$6$}  & \multicolumn{1}{c|}{$7$}  & \multicolumn{1}{c|}{$8$}  & \multicolumn{1}{c|}{$9$}  & \multicolumn{1}{c|}{$10$} & \multicolumn{1}{c|}{$11$} & \multicolumn{1}{c|}{$0$}  & \multicolumn{1}{c|}{$1$}  & \multicolumn{1}{c|}{$2$}  & \multicolumn{1}{c|}{$3$}  & \multicolumn{1}{c|}{$4$}  \\ \cline{2-13} 
\multicolumn{1}{c|}{$6$}  & \multicolumn{1}{c|}{$6$}  & \multicolumn{1}{c|}{$7$}  & \multicolumn{1}{c|}{$8$}  & \multicolumn{1}{c|}{$9$}  & \multicolumn{1}{c|}{$10$} & \multicolumn{1}{c|}{$11$} & \multicolumn{1}{c|}{$0$}  & \multicolumn{1}{c|}{$1$}  & \multicolumn{1}{c|}{$2$}  & \multicolumn{1}{c|}{$3$}  & \multicolumn{1}{c|}{$4$}  & \multicolumn{1}{c|}{$5$}  \\ \cline{2-13} 
\multicolumn{1}{l|}{$7$}  & \multicolumn{1}{c|}{$7$}  & \multicolumn{1}{c|}{$8$}  & \multicolumn{1}{c|}{$9$}  & \multicolumn{1}{c|}{$10$} & \multicolumn{1}{c|}{$11$} & \multicolumn{1}{c|}{$0$}  & \multicolumn{1}{c|}{$1$}  & \multicolumn{1}{c|}{$2$}  & \multicolumn{1}{c|}{$3$}  & \multicolumn{1}{c|}{$4$}  & \multicolumn{1}{c|}{$5$}  & \multicolumn{1}{c|}{$6$}  \\ \cline{2-13} 
\multicolumn{1}{l|}{$8$}  & \multicolumn{1}{c|}{$8$}  & \multicolumn{1}{c|}{$9$}  & \multicolumn{1}{c|}{$10$} & \multicolumn{1}{c|}{$11$} & \multicolumn{1}{c|}{$0$}  & \multicolumn{1}{c|}{$1$}  & \multicolumn{1}{c|}{$2$}  & \multicolumn{1}{c|}{$3$}  & \multicolumn{1}{c|}{$4$}  & \multicolumn{1}{c|}{$5$}  & \multicolumn{1}{c|}{$6$}  & \multicolumn{1}{c|}{$7$}  \\ \cline{2-13} 
\multicolumn{1}{l|}{$9$}  & \multicolumn{1}{c|}{$9$}  & \multicolumn{1}{c|}{$10$} & \multicolumn{1}{c|}{$11$} & \multicolumn{1}{c|}{$0$}  & \multicolumn{1}{c|}{$1$}  & \multicolumn{1}{c|}{$2$}  & \multicolumn{1}{c|}{$3$}  & \multicolumn{1}{c|}{$4$}  & \multicolumn{1}{c|}{$5$}  & \multicolumn{1}{c|}{$6$}  & \multicolumn{1}{c|}{$7$}  & \multicolumn{1}{c|}{$8$}  \\ \cline{2-13} 
\multicolumn{1}{l|}{$10$} & \multicolumn{1}{c|}{$10$} & \multicolumn{1}{c|}{$11$} & \multicolumn{1}{c|}{$0$}  & \multicolumn{1}{c|}{$1$}  & \multicolumn{1}{c|}{$2$}  & \multicolumn{1}{c|}{$3$}  & \multicolumn{1}{c|}{$4$}  & \multicolumn{1}{c|}{$5$}  & \multicolumn{1}{c|}{$6$}  & \multicolumn{1}{c|}{$7$}  & \multicolumn{1}{c|}{$8$}  & \multicolumn{1}{c|}{$9$}  \\ \cline{2-13} 
\multicolumn{1}{l|}{$11$} & \multicolumn{1}{c|}{$11$} & \multicolumn{1}{c|}{$0$}  & \multicolumn{1}{c|}{$1$}  & \multicolumn{1}{c|}{$2$}  & \multicolumn{1}{c|}{$3$}  & \multicolumn{1}{c|}{$4$}  & \multicolumn{1}{c|}{$5$}  & \multicolumn{1}{c|}{$6$}  & \multicolumn{1}{c|}{$7$}  & \multicolumn{1}{c|}{$8$}  & \multicolumn{1}{c|}{$9$}  & \multicolumn{1}{c|}{$10$} \\ \cline{2-13} 
\end{tabular}
\end{table}

\begin{table}[h]
\begin{tabular}{ccccccccccccc}
\textbf{$\cdot$}          & $0$                      & $1$                       & $2$                       & $3$                      & $4$                      & $5$                       & $6$                      & $7$                       & $8$                      & $9$                      & $10$                      & $11$                      \\ \cline{2-13} 
\multicolumn{1}{c|}{$0$}  & \multicolumn{1}{c|}{$0$} & \multicolumn{1}{c|}{$0$}  & \multicolumn{1}{c|}{$0$}  & \multicolumn{1}{c|}{$0$} & \multicolumn{1}{c|}{$0$} & \multicolumn{1}{c|}{$0$}  & \multicolumn{1}{c|}{$0$} & \multicolumn{1}{c|}{$0$}  & \multicolumn{1}{c|}{$0$} & \multicolumn{1}{c|}{$0$} & \multicolumn{1}{c|}{$0$}  & \multicolumn{1}{c|}{$0$}  \\ \cline{2-13} 
\multicolumn{1}{c|}{$1$}  & \multicolumn{1}{c|}{$0$} & \multicolumn{1}{c|}{$1$}  & \multicolumn{1}{c|}{$2$}  & \multicolumn{1}{c|}{$3$} & \multicolumn{1}{c|}{$4$} & \multicolumn{1}{c|}{$5$}  & \multicolumn{1}{c|}{$6$} & \multicolumn{1}{c|}{$7$}  & \multicolumn{1}{c|}{$8$} & \multicolumn{1}{c|}{$9$} & \multicolumn{1}{c|}{$10$} & \multicolumn{1}{c|}{$11$} \\ \cline{2-13} 
\multicolumn{1}{c|}{$2$}  & \multicolumn{1}{c|}{$0$} & \multicolumn{1}{c|}{$2$}  & \multicolumn{1}{c|}{$4$}  & \multicolumn{1}{c|}{$6$} & \multicolumn{1}{c|}{$8$} & \multicolumn{1}{c|}{$10$} & \multicolumn{1}{c|}{$0$} & \multicolumn{1}{c|}{$2$}  & \multicolumn{1}{c|}{$4$} & \multicolumn{1}{c|}{$6$} & \multicolumn{1}{c|}{$8$}  & \multicolumn{1}{c|}{$10$} \\ \cline{2-13} 
\multicolumn{1}{c|}{$3$}  & \multicolumn{1}{c|}{$0$} & \multicolumn{1}{c|}{$3$}  & \multicolumn{1}{c|}{$6$}  & \multicolumn{1}{c|}{$9$} & \multicolumn{1}{c|}{$0$} & \multicolumn{1}{c|}{$3$}  & \multicolumn{1}{c|}{$6$} & \multicolumn{1}{c|}{$9$}  & \multicolumn{1}{c|}{$0$} & \multicolumn{1}{c|}{$3$} & \multicolumn{1}{c|}{$6$}  & \multicolumn{1}{c|}{$9$}  \\ \cline{2-13} 
\multicolumn{1}{c|}{$4$}  & \multicolumn{1}{c|}{$0$} & \multicolumn{1}{c|}{$4$}  & \multicolumn{1}{c|}{$8$}  & \multicolumn{1}{c|}{$0$} & \multicolumn{1}{c|}{$4$} & \multicolumn{1}{c|}{$8$}  & \multicolumn{1}{c|}{$0$} & \multicolumn{1}{c|}{$4$}  & \multicolumn{1}{c|}{$8$} & \multicolumn{1}{c|}{$0$} & \multicolumn{1}{c|}{$4$}  & \multicolumn{1}{c|}{$8$}  \\ \cline{2-13} 
\multicolumn{1}{c|}{$5$}  & \multicolumn{1}{c|}{$0$} & \multicolumn{1}{c|}{$5$}  & \multicolumn{1}{c|}{$10$} & \multicolumn{1}{c|}{$3$} & \multicolumn{1}{c|}{$8$} & \multicolumn{1}{c|}{$1$}  & \multicolumn{1}{c|}{$6$} & \multicolumn{1}{c|}{$11$} & \multicolumn{1}{c|}{$4$} & \multicolumn{1}{c|}{$9$} & \multicolumn{1}{c|}{$2$}  & \multicolumn{1}{c|}{$7$}  \\ \cline{2-13} 
\multicolumn{1}{c|}{$6$}  & \multicolumn{1}{c|}{$0$} & \multicolumn{1}{c|}{$6$}  & \multicolumn{1}{c|}{$0$}  & \multicolumn{1}{c|}{$6$} & \multicolumn{1}{c|}{$0$} & \multicolumn{1}{c|}{$6$}  & \multicolumn{1}{c|}{$0$} & \multicolumn{1}{c|}{$6$}  & \multicolumn{1}{c|}{$0$} & \multicolumn{1}{c|}{$6$} & \multicolumn{1}{c|}{$0$}  & \multicolumn{1}{c|}{$6$}  \\ \cline{2-13} 
\multicolumn{1}{c|}{$7$}  & \multicolumn{1}{c|}{$0$} & \multicolumn{1}{c|}{$7$}  & \multicolumn{1}{c|}{$2$}  & \multicolumn{1}{c|}{$9$} & \multicolumn{1}{c|}{$4$} & \multicolumn{1}{c|}{$11$} & \multicolumn{1}{c|}{$6$} & \multicolumn{1}{c|}{$1$}  & \multicolumn{1}{c|}{$8$} & \multicolumn{1}{c|}{$3$} & \multicolumn{1}{c|}{$10$} & \multicolumn{1}{c|}{$5$}  \\ \cline{2-13} 
\multicolumn{1}{c|}{$8$}  & \multicolumn{1}{c|}{$0$} & \multicolumn{1}{c|}{$8$}  & \multicolumn{1}{c|}{$4$}  & \multicolumn{1}{c|}{$0$} & \multicolumn{1}{c|}{$8$} & \multicolumn{1}{c|}{$4$}  & \multicolumn{1}{c|}{$0$} & \multicolumn{1}{c|}{$8$}  & \multicolumn{1}{c|}{$4$} & \multicolumn{1}{c|}{$0$} & \multicolumn{1}{c|}{$8$}  & \multicolumn{1}{c|}{$4$}  \\ \cline{2-13} 
\multicolumn{1}{c|}{$9$}  & \multicolumn{1}{c|}{$0$} & \multicolumn{1}{c|}{$9$}  & \multicolumn{1}{c|}{$6$}  & \multicolumn{1}{c|}{$3$} & \multicolumn{1}{c|}{$0$} & \multicolumn{1}{c|}{$9$}  & \multicolumn{1}{c|}{$6$} & \multicolumn{1}{c|}{$3$}  & \multicolumn{1}{c|}{$0$} & \multicolumn{1}{c|}{$9$} & \multicolumn{1}{c|}{$6$}  & \multicolumn{1}{c|}{$3$}  \\ \cline{2-13} 
\multicolumn{1}{c|}{$10$} & \multicolumn{1}{c|}{$0$} & \multicolumn{1}{c|}{$10$} & \multicolumn{1}{c|}{$8$}  & \multicolumn{1}{c|}{$6$} & \multicolumn{1}{c|}{$4$} & \multicolumn{1}{c|}{$2$}  & \multicolumn{1}{c|}{$0$} & \multicolumn{1}{c|}{$10$} & \multicolumn{1}{c|}{$8$} & \multicolumn{1}{c|}{$6$} & \multicolumn{1}{c|}{$4$}  & \multicolumn{1}{c|}{$2$}  \\ \cline{2-13} 
\multicolumn{1}{c|}{$11$} & \multicolumn{1}{c|}{$0$} & \multicolumn{1}{c|}{$11$} & \multicolumn{1}{c|}{$10$} & \multicolumn{1}{c|}{$9$} & \multicolumn{1}{c|}{$8$} & \multicolumn{1}{c|}{$7$}  & \multicolumn{1}{c|}{$6$} & \multicolumn{1}{c|}{$5$}  & \multicolumn{1}{c|}{$4$} & \multicolumn{1}{c|}{$3$} & \multicolumn{1}{c|}{$2$}  & \multicolumn{1}{c|}{$1$}  \\ \cline{2-13} 
\end{tabular}
\end{table}
\end{center}
\end{solution}
%\newpage
 \begin{exercicio}
 Busque os inversos dos seguintes elementos:
\itensladoalado{4}{
	\task[\alt{a}] $\overline{14}$ em $\mathbb Z_{15}$;
	\task[\alt{b}] $\overline{38}$ em $\mathbb Z_{83}$;
\task[\alt{c}] $\overline{351}$ em $\mathbb Z_{6669}$;
\task[\alt{d}] $\overline{91}$ em $\mathbb Z_{2565}$.
}
 \end{exercicio}
 \begin{solution}
 \itens{
 \task[\alt{a}] Temos que
      $$\begin{array}{rcl}
          \textcolor{JungleGreen}{14}x&\equiv&1 \pmod{15}\\
     \textcolor{JungleGreen}{-1} \cdot x&\equiv&1 \pmod{15}\\
     x&\equiv&\textcolor{Plum}{-1} \pmod{15}\\
     x&\equiv&\textcolor{Plum}{14} \pmod{15}\\
     \overline{14}^{-1}&=&\overline{14}.
     \end{array}$$
 \task[\alt{b}] Temos que
      $$\begin{array}{rcl}
        38x&\equiv&1 \pmod{83}\\
       \textcolor{Cyan}{2} \cdot 38x&\equiv& \textcolor{Cyan}{2} \cdot 1 \pmod{83}\\
     \textcolor{JungleGreen}{76}x&\equiv&2 \pmod{83}\\
     \textcolor{JungleGreen}{-7}x&\equiv&2  \pmod{83}\\
         \textcolor{Magenta}{12} \cdot (-7)x&\equiv& \textcolor{Magenta}{12} \cdot 2  \pmod{83}\\
     12\cdot(-7)x&\equiv&24  \pmod{83}\\
     \textcolor{Plum}{-84}x&\equiv&24  \pmod{83}\\
     \textcolor{Plum}{-1} \cdot x&\equiv&24  \pmod{83}\\
     x&\equiv&\textcolor{Orange}{-24}  \pmod{83}\\
     x&\equiv&\textcolor{Orange}{59}  \pmod{83}\\
     \overline{38}^{-1}&=&\overline{59}
     \end{array}
     $$
      \task[\alt{c}] Como $\mdc(351,6669)=3\neq 1$, então nesse caso não existe $\overline{351}^{-1}.$
     \task[\alt{d}] Temos que
      $$\begin{array}{rcl}
        91x&\equiv&1 \pmod{2565}\\
 \textcolor{Cyan}{28} \cdot 91x &\equiv& \textcolor{Cyan}{28} \cdot1 \pmod{2565} \\
 \textcolor{JungleGreen}{2548}x &\equiv& 28 \pmod{2565} \\
  \textcolor{JungleGreen}{-17}x &\equiv& 28 \pmod{2565} \\
  17x &\equiv & -28 \pmod{2565} \\
   \textcolor{Magenta}{151} \cdot 17x &\equiv & \textcolor{Magenta}{151} \cdot(-28) \pmod{2565} \\
   \textcolor{Plum}{2567}x &\equiv& -4228 \pmod{2565} \\
   \textcolor{Plum}{2}x &\equiv& \textcolor{Orange}{-4228} \pmod{2565} \\
2x &\equiv& \textcolor{Orange}{902} \pmod{2565} \\
\textcolor{Emerald}{1283} \cdot 2x &\equiv& \textcolor{Emerald}{1283} \cdot 902 \pmod{2565} \\
\textcolor{Brown}{2566}x &\equiv & 1157266 \pmod{2565} \\
\textcolor{Brown}{1}x &\equiv & \textcolor{NavyBlue}{1157266} \pmod{2565} \\
x &\equiv & \textcolor{NavyBlue}{451} \pmod{2565} \\
     \overline{91}^{-1}&=&\overline{451}.
    \end{array}$$
    \textcolor{Red}{Solução Alternativa:} Pelo algoritmo de Euclides, temos:
     $$1=451\cdot91-16\cdot2565$$
     $$\overline{1}=\overline{451}\cdot\overline{91}$$
     $$\overline{91}^{-1}=\overline{451}$$
 }
 
 \end{solution}
 \begin{exercicio}
 Mostre que
 \itensladoalado{4}{
 	\task[\alt{a}] $\overline{73}=\overline{-92}$ em $\mathbb Z_{5}$;
	\task[\alt{b}] $\overline{99}=\overline{-87}$ em $\mathbb Z_{6}$;
	\task[\alt{c}] $\overline{3!}=\overline{-2!}$ em $\mathbb Z_{8}$;
	\task[\alt{d}] $\overline{12!}=\overline{15!}$ em $\mathbb Z_{9}$.
 
 }
 \end{exercicio}
 \begin{solution}
 
 \end{solution}
 
 \begin{exercicio}
 Em $\mathbb Z_{20}$, determine
\itens{
	\task[\alt{a}] os menores representantes positivos de $\overline{-10}$ e $\overline{-6}$;
	\task[\alt{b}]todos os divisores de zero;
	\task[\alt{c}] todos os elementos inversos com seus inversos;
	\task[\alt{d}] repita os itens (b) e (c) para $\mathbb Z_{10}$ e $\mathbb Z_{12}$.
	}
 \end{exercicio}
 \begin{solution}
 
 \end{solution}
 \begin{exercicio}
 Determine os inversos multiplicativos de $\overline{a}$ em $\mathbb Z_n$ e, em seguida, resolva as equações de congruências reduzidas:
\itens{
\task[\alt{a}] $a = 3, \mathbb Z_n = \mathbb Z_{10}$ e $3x\equiv 7 \pmod{10}$;
\task[\alt{b}] $a = 6, \mathbb Z_n = \mathbb Z_{35}$ e $6x-2\equiv 11 \pmod{35}$.
}
 \end{exercicio}
  \begin{solution}
 
 \end{solution}
 
 \begin{exercicio}
 Sejam $\overline{a},\overline{b},\overline{c} \in \mathbb Z_m$ com $\mdc(c,m)=1$. Prove que $\overline{a} \cdot \overline{c}=\overline{b} \cdot \overline{c}$ implica que $\overline{a}=\overline{b}$.
 \end{exercicio}
 \begin{solution}
  
 \end{solution}
 \begin{exercicio}
 Sejam $p$ um primo e $\overline{a},\overline{b} \in \mathbb Z_p$. Prove que
\itens{
	\task[\alt{a}] $\overline{a}^p=\overline{a}$;
\task[\alt{b}]  $(\overline{a}+\overline{b})^p=\overline{a}+\overline{b}$.}
 \end{exercicio}
 
 \begin{solution}
  
 \end{solution}
 \newpage
\begin{exercicio}
O elemento $\overline{a} \in \mathbb Z_m$ chama-se \textbf{idempotente} se $\overline{a}\cdot \overline{a}=\overline{a}$.
\itens{
	\task[\alt{a}] Busque todos idempotentes em $\mathbb Z_6$ e $\mathbb Z_{12}$.
	\task[\alt{a}] Busque todos idempotentes em $\mathbb Z_{10}$ e $\mathbb Z_{30}$.
	\task[\alt{c}] Seja $p$ um primo. Mostre que $\overline{0}, \overline{1}$ são os únicos idempotentes em $\mathbb Z_p$.
	}
\end{exercicio} 
 \begin{solution}
  
 \end{solution}
 \begin{exercicio}
 O elemento $\overline{a} \in \mathbb Z_m$ chama-se \textbf{nilpotente} se $\overline{a}^{k}=\overline{0}$ para algum $k$. Mostre que $\mathbb Z_m$ não tem não-nulos nilpotentes se e só se $m$ não tem fator primo em quadrado. 
 \end{exercicio}
 \begin{solution}
  
 \end{solution}

\begin{exercicio}
Em $\mathbb Z_{7}$, busque os quadrados de todos elementos. \label{ex1}
\end{exercicio} 
\begin{solution}
 Temos que
  $$\overline{0}\cdot\overline{0}=\overline{0}$$
 $$\overline{1}\cdot\overline{1}=\overline{1}$$
 $$\overline{2}\cdot\overline{2}=\overline{4}$$
 $$\overline{3}\cdot\overline{3}=\overline{2}$$
 $$\overline{4}\cdot\overline{4}=\overline{2}$$
 $$\overline{5}\cdot\overline{5}=\overline{4}$$
 $$\overline{6}\cdot\overline{6}=\overline{1}$$
\end{solution}


\begin{exercicio}
Encontre as raízes em $\mathbb Z_{7}$ de 
\itensladoalado{2}{

\task[\alt{a}] $x^2 + x+ \overline{1}$ 
\task[\alt{b}] $\overline{3}x^2 + \overline{4}x+ \overline{3}$
}
por completar o quadrado e usando Exercício \ref{ex1}.
\end{exercicio}
\begin{solution}
\itens{
\task[\alt{a}] Por inspeção, usando o Exercício 10, temos $S=\left\{\overline{2},\overline{4}\right\}$, pois $(\overline{2}+\overline{1})\cdot(\overline{2}+\overline{1})=\overline{2}$ e $(\overline{4}+\overline{1})\cdot(\overline{4}+\overline{1})=\overline{4}$.
 \task[\alt{b}] Novamente, utilizando o Exercício 10,
$$\begin{array}{rcl}
    \overline{3}x^2 + \overline{4}x+ \overline{3}&=&\overline{0}\\
    \overline{5}\cdot\left(\overline{3}x^2 + \overline{4}x+ \overline{3}\right)&=&\overline{5}\cdot\overline{0}\\
    x^2 + \overline{6}x+ \overline{1}&=&\overline{0}\\
    \left(x+ \overline{3}\right)^2&=&\overline{1} 
 \end{array}$$
Por inspeção, $S=\left\{\overline{3},\overline{5}\right\}$, pois, pelo Exercício 10, temos $(\overline{x}+\overline{3})=\overline{1}$ ou $(\overline{x}+\overline{3})=\overline{6}$.
 }
\end{solution}

\begin{exercicio}
Encontre os quadrados de todos elementos em $\mathbb{Z}_{11}.$ \label{ex2}
\end{exercicio}
\begin{solution}
Temos
 $$\overline{0}\cdot\overline{0}=\overline{0}$$
 $$\overline{1}\cdot\overline{1}=\overline{1}$$
 $$\overline{2}\cdot\overline{2}=\overline{4}$$
 $$\overline{3}\cdot\overline{3}=\overline{9}$$
 $$\overline{4}\cdot\overline{4}=\overline{5}$$
 $$\overline{5}\cdot\overline{5}=\overline{3}$$
 $$\overline{6}\cdot\overline{6}=\overline{3}$$
 $$\overline{7}\cdot\overline{7}=\overline{5}$$
 $$\overline{8}\cdot\overline{8}=\overline{9}$$
 $$\overline{9}\cdot\overline{9}=\overline{4}$$
 $$\overline{10}\cdot\overline{10}=\overline{1}$$
\end{solution}
\begin{exercicio}
Encontre as raízes em $\mathbb Z_{11}$ de 
\itensladoalado{2}{
\task[\alt{a}] $\overline{4}x^2 + \overline{6}x+ \overline{1}$
\task[\alt{b}] $\overline{4}x^2 + \overline{6}x+ \overline{8}$
}
por completar o quadrado e usando Exercício \ref{ex2}.
\end{exercicio}
\begin{solution}
\itens{
\task[\alt{a}] Observe que
$$\begin{array}{rcl}
    \overline{4}x^2 + \overline{6}x+ \overline{1}&=&\overline{0}\\
    \overline{3}\cdot\left(\overline{4}x^2 + \overline{6}x+ \overline{1}\right)&=&\overline{3}\cdot\overline{0}\\
    x^2 + \overline{7}x+ \overline{3}&=&\overline{0}\\
    x^2 + \overline{6}x+ \overline{9}&=&-x+\overline{6}\\
    \left(x+ \overline{3}\right)^2&=&-x+\overline{6} 
\end{array}$$
Por inspeção, concluímos que $S=\left\{\overline{1},\overline{3}\right\}$.
\task[\alt{b}] Temos 
    $$\begin{array}{rcl}
    \overline{4}x^2 + \overline{6}x+ \overline{8}&=&\overline{0}\\
    \overline{3}\cdot\left(\overline{4}x^2 + \overline{6}x+ \overline{8}\right)&=&\overline{3}\cdot\overline{0}\\
    x^2 + \overline{7}x+ \overline{2}&=&\overline{0}\\
    x^2 + \overline{6}x+ \overline{9}&=&-x+\overline{7}\\
    \left(x+ \overline{3}\right)^2&=&-x+\overline{7} 
\end{array}$$
Por inspeção, $S=\emptyset$, ou seja, a equação não possui soluções em $\mathbb{Z}_{11}.$ 
}
\end{solution}

\begin{exercicio}
Determine os divisores de zero, em $\mathbb Z_m$, e resolva as equações para cada caso:
\itensladoalado{2}{
    \task[\alt{a}] $\overline{7}x=\overline{0}$, $m=21$;
   \task[\alt{b}] $\overline{4}x=\overline{10}$, $m=22$;  
      \task[\alt{c}]     
    $\overline{3}x=\overline{6}$, $m=24$;  
   \task[\alt{d}]   
    $\overline{5}x=\overline{0}$, $m=25$;.}
\end{exercicio}
\begin{solution}

\end{solution}
\begin{exercicio}
Encontre os divisores de zero, em $\mathbb Z_m$, para $m=8,9,10,14,15,26,28$.
\end{exercicio}
\begin{solution}
 Em $\mathbb{Z}_8: \overline{2}, \overline{4}, \overline{6}$.
 
 Em $\mathbb{Z}_9: \overline{3}, \overline{6}$.
 
 Em $\mathbb{Z}_{10}: \overline{2},\overline{4}, \overline{5}, \overline{6}, \overline{8}$.
 
 Em $\mathbb{Z}_{14}: \overline{2},\overline{4}, \overline{6}, \overline{7}, \overline{8}, \overline{10}, \overline{12}$.
 
 Em $\mathbb{Z}_{15}: \overline{3},\overline{5},\overline{6}, \overline{9}, \overline{10}, \overline{12}$.
 
 Em $\mathbb{Z}_{26}: \overline{2},\overline{4}, \overline{6}, \overline{8}, \overline{10}, \overline{12}, \overline{13}, \overline{14}, \overline{16}, \overline{18}, \overline{20}, \overline{22},\overline{24}$.
 
 Em $\mathbb{Z}_{28}: \overline{2},\overline{4}, \overline{6},\overline{7}, \overline{8}, \overline{10}, \overline{12}, \overline{13}, \overline{14}, \overline{16}, \overline{18}, \overline{20},\overline{21}, \overline{22}, \overline{24}, \overline{26}$.
\end{solution}

\begin{exercicio}
Ache os divisores de zero e os elementos que tem inversos em $\mathbb Z_5$, $\mathbb Z_8$, $\mathbb Z_{17}$, $\mathbb Z_{21}$  e $\mathbb Z_{89}$.
\end{exercicio}
\begin{solution}
 Em $\mathbb{Z}_{5}, \mathbb{Z}_{17}$ e $\mathbb{Z}_{89}$ não há divisores de zero e todos os elementos são inversíveis.
 
 Em $\mathbb{Z}_{8}$ os divisores de zero são $ \overline{2}, \overline{4}, \overline{6}$ e os elementos inversíveis são $ \overline{1}, \overline{3}, \overline{5}$ e $\overline{7}$.
 
 Em $\mathbb{Z}_{21}$ os divisores de zero são $ \overline{3}, \overline{6}, \overline{7},\overline{9}, \overline{12}, \overline{14},\overline{15}$ e $ \overline{18}$ e os elementos inversíveis são $ \overline{1}, \overline{2}, \overline{4},\overline{5}, \overline{8}, \overline{10},\overline{11}, \overline{13}, \overline{16}, \overline{17}, \overline{19}$ e  $\overline{20}$.
\end{solution}

\begin{exercicio}
Resolva, em $\mathbb Z_m$, as equações abaixo:
\itensladoalado{2}{
\task[\alt{a}] $\overline{3}x+\overline{2}=\overline{6}x+\overline{7}$, $m=8$;
\task[\alt{b}] $(\overline{2}x+\overline{3})^2+(\overline{3}x+\overline{2})^2+\overline{5}x=\overline{0}$, $m=5$;
\task[\alt{c}] $\overline{4}x-\overline{7}+\overline{6}x+\overline{2}=\overline{3}x+\overline{5}x$, $m=12$;
\task[\alt{d}] $x^{21}-x=\overline{0}$, $m=5$;
\task[\alt{e}] $x^{12}-\overline{1}=\overline{0}$, $m=5$;
\task[\alt{f}] $x^{7}-x=\overline{0}$, $m=4$.
}
\end{exercicio}
\begin{solution}

\end{solution}

\begin{exercicio}
 Resolva em $\mathbb Z_m$ cada um dos sistemas abaixo:
 \itensladoalado{2}{
\task[\alt{a}] $
\left \{ \begin{array}{r c l}
	\overline{4}x+y&=&\overline{1}\\
	x-\overline{2}y&=&\overline{4}.
	\end{array} \right. , m = 5$
\task[\alt{b}] 	$\left \{ \begin{array}{r c l}
	x+y+z&=&\overline{0}\\
	\overline{2}x+\overline{3}y+\overline{3}z&=&\overline{3}\\
	x+y+\overline{3}z&=&\overline{0}.	
	\end{array} \right. , m = 4$
	}
\end{exercicio}
\begin{solution}

\end{solution}

\newpage
\begin{exercicio}
 Verifique se os elementos abaixo são inversíveis. Em caso afirmativo, determine o inverso.
\itensladoalado{2}{
	\task[\alt{a}] $\overline{97}$ em $\mathbb Z_{307}$;
	\task[\alt{b}] $\overline{22}$ em $\mathbb Z_{105}$.
	}
\end{exercicio}
\begin{solution}

\end{solution}

\begin{exercicio}
 Seja $p$ um número primo. Prove que $\overline{2}, \overline{3}, \ldots, \overline{p-1}$ são soluções em $\mathbb{Z}_p$ da equação
 \[
 x^{p-2} + x^{p-3} + \ldots + x + \overline{1} = \overline{0}.
 \]
 \textsf{[Dica:]} Utilize a fatoração $x^{p-1} - 1 = (x-1)(x^{p-2} + x^{p-3} + \ldots + x + 1).$
\end{exercicio}
\begin{solution}
 Observe que 
 \[(x^{p-1} - 1) = (x-1)(x^{p-2} + x^{p-3} + \ldots + x + 1)\] 
 Considerando essa identidade em $\mathbb{Z}_p,$ como $p$ é primo, existe um $\overline{x-1}^{-1} \in \mathb{Z}_p$ tal que $\overline{x-1} \cdot \overline{x-1}^{-1} = \overline{1}.$ Pelo Pequeno Teorema de Fermat, sabemos que $x^{p-1} \equiv 1 \pmod p,$ para todo $x$ tal que $\mdc(x, p) = 1,$ ou seja, em $\mathbb{Z}_p,$ $x= \overline{2}, \ldots, \overline{p-1}$. Além disso, $\overline{x^{p-1} - 1} = \overline{1 - 1} = \overline{0}.$ Logo,
 \[\textcolor{Green}{\overline{(x^{p-1} - 1)}} = \overline{(x-1)}\overline{(x^{p-2} + x^{p-3} + \ldots + x + 1)} \Rightarrow \overline{(x-1)}^{-1}\textcolor{Green}{\overline{0}} = \overline{(x^{p-2} + x^{p-3} + \ldots + x + 1)} \Rightarrow  x^{p-2} + x^{p-3} + \ldots + x + \overline{1} = \overline{0}
 \]
 para todo $x= \overline{2}, \ldots, \overline{p-1}.$
\end{solution}
\begin{exercicio}
 A \textbf{ordem} de um elemento $\overline{a}$ em $\mathbb{Z}_p$ é definida como o menor inteiro positivo $m$ tal que $\overline{a}^m = \overline{1}.$
 \itens{
 \task[\alt{a}] Prove que $m \le p-1.$
 \task[\alt{b}] Encontre as ordens de todos os elementos de $\mathbb{Z}_{11}$ e $\mathbb{Z}_{13}.$
 }
\end{exercicio}
\begin{solution}
 \itens{
 \task[\alt{a}] Pelo Pequeno Teorema de Fermat, temos que
 \[a^{p-1} \equiv 1 \pmod p,\]
 ou seja, $\overline{a}^{p-1} = \overline{1}.$ Assim, $m \le p-1.$

 \task[\alt{b}] Seguem as tabelas:%https://web.evanchen.cc/handouts/ORPR/ORPR.pdf
 }
\end{solution}
 \end{document}